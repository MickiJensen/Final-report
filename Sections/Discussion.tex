\section{Discussion}
Things to discuss
\begin{itemize}
  \item Which implications does it have to set V equal to a constant. Read page 3 (says that it is a measure of functional diversity).
  \item Explain the whole immigration term (see explanation in \cite{AcevedoTrejos2015} of why they included it. Sounds a bit similar to our issue.
  \item Maybe mention that some of the assumptions that we have done only work for tropical regions, and not for places with lot of seasonality
\end{itemize}

\subsection{Criticism of model made by  Acevedo-Trejoset al. (2015)}
One of the main reasons why we became interested in the trait-based plankton model by \cite{AcevedoTrejos2015}, was because of their simplistic way of modelling some of the most important trade-offs of related to the cell size of phytoplankton. Despite their simple formulations, the model still managed to capture some overall characteristics of how the mean cell size of a phytoplankton community is influence under different environmental conditions. A simple reconstruction of the model however, turned out to be more challenging to implement than we first expected. This was mainly due to the lack of a more comprehensive dissemination of their methodology and their underlying model assumptions. In the paper, they for instance state that they log-transform the size trait, without mentioning whether that is in terms of a natural logarithmic or log10 scale. Moreover, they’re results are presented with the cell size having the units of log(mean cell size), which results in a poor understanding of their results in general. 
%The trait-based model presented in the paper by \cite{Acevedo-Trejos2015} was interesting to work with as their implistic
%The trait-based model is relatively simple to understad. At the same time, they capture many of the
%How did they find the Influence on cell side
%underlying model assumptions, conversion between units and ininformation regarding
